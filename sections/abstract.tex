\section*{Abstract}
Für das präzise Ansteuern eines Alexandrit Femtosekunden Lasers, der breit in der Industrie und der Wissenschaft und Forschung eingesetzt werden soll, wurde eine Steuerung gebaut. Diese soll es ermöglichen geeignete Arbeitspunkte zu finden, in denen der Laser betreiben werden soll. Die Temperatur der Pumpdiode und des Kristalls müssen sich im Bereich von 18°C bzw. 20°C befinden, wo der Laser ideal läuft. Die Einhaltung der Temperaturen wird mit thermoelektrischen Elementen (TEC) erreicht, die von einem TEC-Kontroller geregelt werden. Die Ausgangsleistung des Lasers wird mit einem Diodentreiber über die Variation des Stromflusses gesteuert. Die zentrale Steuereinheit bildet ein einziger Rechner. Hier wurde ein Raspberry PI der Generation 3B+ eingesetzt. Für die Steuerung des Diodentreibers benötigte es noch analoge Ein- und Ausgänge, die von einer SPS zur Verfügung gestellt wurden. Alle zum Betrieb der Steuerung benötigten Komponenten, wie Energieversorgung, Diodentreiber, TEC-Kontroller, und Rechner wurden in einem kompakten Gehäuse untergebracht. Auf Grund von Auslaufen der Zeit, konnten nicht alle Funktionen implementiert werden. Dies hat zur Folge, dass die Steuerung nicht in Betrieb genommen werden konnte. Dies hatte auch zur Folge, dass die Optimierung der Glasfaser nicht in Angriff genommen werden konnte.

% In einem Abstract <<Key-words>> einbauen
% in einem mgt summary gibt es Kapitel wohingegen das Abstract nur ein Paragraph ist

\clearpage
\section*{Vorwort}
Kap. 1 beschreibt die Idee und die Voraussetzungen für den Aufbau der Steuerung des Alexandrit-Femtosekunden Laser. Darin beschreiben sind bereits Komponenten, die im Eingangs vermutlich geeignet für die Steuerung gewesen sind. Teilweise wurden diese im Verlauf der Arbeit jedoch abgeändert. In Kap. 2 werden die Grundlagen für das Verständnis der folgenden Kapitel hauptsächlich im Bereich der Softwareentwicklung und Struktur von Computerprogrammen vermittelt. Kap. 3 beinhaltet die genauere Beschreibung der Komponenten, die in der Steuerung schlussendlich verwendet wurden. In Kap. 4 wird die Struktur der Software erläutert und die Feinheiten, die die Software aufweist. Im Vordergrund stehen wie mit den produzierten Daten umgegangen wird. Kap. 5 beschreibt kurz den Testaufbau der verwendet wurde um die Steuerung auf ihre Funktion zu testen, solange dies im Labor der FHNW nicht möglich gewesen ist. Im letzten Kap. 6 werden nochmals auf die Steuerung eingegangen. Im Vordergrund stehen die Erkenntnisse zur Verbesserung und der Vollendung der Steuerung.

\section*{Danksagungen}
An dieser Stelle möchte ich mich bei all denjenigen bedanken, die mich in dieser Bachelorarbeit Betreut und Unterstützt haben.

Als erstes möchte ich mich bei Prof. Dr. Bojan Resan für das bereit gestellte Wissen, die Betreuung, die konstruktive Kritik und hilfreichen Anregungen an meiner Arbeit bedanken.\\

Zusätzlich möchte ich für die Unterstützung weiteren Betreuenden Personen danken, ohne die diese Projektarbeit nicht durchführbar gewesen wäre.\\ Bei Tobias Grätzer für die Einführung am Alexandrit-Laser im Labor und der Lektüre und dem Material bezüglich Lasertechnik von denen ich sehr viel profitieren konnte.

Daneben geht mein Dank an Romain Caretto, Philip Burger und Daniel Hug vom IPPE, welche mir Komponenten und Erfahrungen zur Verfügung gestellt haben, damit ich meine Steuerung auf ihre Funktion testen konnte.\\

Zum Schluss möchte ich mich bei meinem Arbeitskollegen Marko Obradovic für das Reviewen meines Programmcodes und den Verbesserungsvorschlägen dessen bedanken.
\\\\
Leroy Harreh\\
Basel, März 2024

