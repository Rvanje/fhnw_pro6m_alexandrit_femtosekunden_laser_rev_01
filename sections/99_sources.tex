$[1]$ T. Grätzer, «Supervisor: Prof. Dr. Bojan Resan Student: Tobias Grätzer».\\

$[2]$	I. Sommerville, Software Engineering, 9. Aufl. Hallbergermoos Deutschland: Pearson Deutschland GmbH, 2012.\\

$[3]$	Erich Gamma, John Vlissides, Richard Helm, und Ralph Johnson, Design Patterns Elements of Reusable Object-Oriented Software. \\

$[4]$	«Framework», Wikipedia. 21. September 2023. Zugegriffen: 25. Oktober 2023. [Online].\\
Verfügbar unter: https://de.wikipedia.org/w/index.php?title=Framework\&oldid=237530238\\

$[5]$	Peter Mandl, Grundkurs Betriebssysteme Architekturen, Betreibsmittelverwaltung, Synchronisation, Prozesskommunikation, Virtualisierung, 5. Aufl. Springer Vieweg.\\

$[6]$ H. Maurice und Nir Shavait, The Art of Multiprocessor Programming, 1. Aufl. USA: Elsevier, 2012.\\

$[7]$	«So prüfen und überwachen Sie die Temperatur der CPU», So prüfen und überwachen Sie die Temperatur der CPU. Zugegriffen: 20. März 2024. [Online].\\
Verfügbar unter: https://www.avast.com/de-de/c-how-to-check-cpu-temperature\\

$[8]$	L. Harreh, «Projektbeschreibung Entwicklung der Steuerung für einen Alexandrit-Femto- sekunden Lasers», . November, 2023.\\

$[9]$	 Reichelt elektronik G. I. Team (webmaster@reichelt.de), «MW RSP-200-7.5 - Schaltnetzteil, geschlossen, 200 W, 7,5 V, 26,7 A», Elektronik und Technik bei reichelt elektronik günstig bestellen.\\
Zugegriffen: 20. März 2024. [Online].\\
Verfügbar unter: https://www.reichelt.com/ch/de/schaltnetzteil-geschlossen-200-\\ w-7-5-v-26-7-a-mw-rsp-200-7-5-p185832.html\\

$[10]$	«pandas (Software)», Wikipedia. 28. Oktober 2022.\\
Zugegriffen: 20. März 2024. [Online].\\
Verfügbar unter: https://de.wikipedia.org/w/index.php?title=Pandas\_(Software)\&\\oldid=227426571\\

$[11]$	«NumPy», Wikipedia. 20. Februar 2024.\\
Zugegriffen: 20. März 2024. [Online].\\
Verfügbar unter: https://en.wikipedia.org/w/index.php?title=NumPy\&oldid=1209243783\\

$[12]$	J. Ernesti und P. Kaiser, Python 3: das umfassende Handbuch; [Einstieg, Praxis, Referenz; Sprachgrundlagen, Objektorientierung, Modularisierung; Migration, Debugging, Interoperabilität mit C, GUIs, Netzwerkkommunikation u.v.m.], 3., Aktualis. u. erw. Aufl. 2012, 2., korr. Nachdr. 2014. in Galileo Computing. Bonn: Galileo Press, 2014.\\

$[13]$	«Universal Serial Bus», Wikipedia. 8. März 2024.\\
Zugegriffen: 20. März 2024. [Online].\\
Verfügbar unter: https://de.wikipedia.org/w/index.php?\\ title=Universal\_Serial\_Bus\&oldid=242923181\\

$[14]$	«Laser Diode Driver CW 15А 10V SF6015», maiman.\\
Zugegriffen: 20. März 2024. [Online].\\
Verfügbar unter: https://www.maimanelectronics.com/product\\
-page/laser-diode-driver-cw-15a-10v-sf6015\\

$[15]$ R. P. Ltd, «Buy a Raspberry Pi 4 Model B», Raspberry Pi.\\
Zugegriffen: 20. März 2024. [Online].\\
Verfügbar unter: https://www.raspberrypi.com/products/raspberry-pi-4-model-b/\\

$[16]$ V. Technologies, «An Introduction to Asynchronous Programming in Python», Velotio Perspectives. Zugegriffen: 20. März 2024. [Online]. Verfügbar unter: https://medium.com/velotio-perspectives/an-introduction-to-asynchronous-programming-in-python-af0189a88bbb\\

$[17]$ «The Difference Between Asynchronous and Multi-Threading | Baeldung on Computer Science».\\
Zugegriffen: 20. März 2024. [Online].\\
Verfügbar unter: https://www.baeldung.com/cs/async-vs-multi-threading

$[18]$ «Raspberry Pi – Kontron Electronics - Kontron Electronics».\\
Zugegriffen: 20. März 2024. [Online].\\
Verfügbar unter: https://www.kontron-electronics.de/produkte/raspberry-pi/

$[19]$	1.1.0 Thermoelectric Cooling, (14. August 2017).\\
Zugegriffen: 23. September 2023. [Online Video].\\
Verfügbar unter: https://www.youtube.com/watch?v=kzauiUH5ohc

$[20]$	L. Harreh, «Konstruktion der Kristallhalterung für den Alexandrit Femtosekunden Laser», 2023.

$[21]$	«Opt Lasers LPLDD-1,5A-35V-TP w/ Thermal Protection Circuit».\\
Zugegriffen: 20. März 2024. [Online].\\
Verfügbar unter: https://optlasers.com/laser-diode-\\
drivers/lpldd-1500ma-35v-tp-laser-diode-driver

$[22]$ Z. Xiao-Guang und Z. Yuan, «The number of least degrees of freedom required for a polarization controller to transform any state of polarization to any other output covering the entire Poincaré sphere», Chin. Phys. B, Bd. 17, Nr. 7, S. 2509–2513, Juli 2008, doi: 10.1088/1674-1056/17/7/027.

$[23]$  dsysd dev, «System Design Patterns: Producer Consumer Pattern», Medium.\\
Zugegriffen: 21. März 2024. [Online].\\
Verfügbar unter: https://dsysd-dev.medium.com/system-design-patterns-producer-consumer\\
-pattern-1572f813329b

$[24]$  «CP604060395 CUI Devices | Mouser», Mouser Electronics.\\
Zugegriffen: 27. April 2024. [Online].\\
Verfügbar unter: https://www.mouser.ch/ProductDetail/CUI-Devices/CP604060395?qs=Cb2nCFKsA8rD9rJyfrs6oA\%3D\%3D

$[25]$  «Diode Laser\_Products\_BWT».\\
Zugegriffen: 27. April 2024. [Online].\\
Verfügbar unter: https://www.bwt-bj.com/en/product/