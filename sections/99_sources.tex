$[1]$	A. Rohrbacher, O. E. Olarte, V. Villamaina, P. Loza-Alvarez, und B. Resan, «Multiphoton imaging with blue-diode-pumped SESAM-modelocked Ti:sapphire oscillator generating 5 nJ 82 fs pulses», Opt. Express, Bd. 25, Nr. 9, S. 10677, Mai 2017, doi: 10.1364/OE.25.010677.\\

$[2]$	S. Ghanbari, K. A. Fedorova, A. B. Krysa, E. U. Rafailov, und A. Major, «Femtosecond Alexandrite laser passively mode-locked by an InP/InGaP quantum-dot saturable absorber», Opt. Lett., Bd. 43, Nr. 2, S. 232, Jan. 2018, doi: 10.1364/OL.43.000232.\\

$[3]$   S. Labisch und G. Wählisch, Technisches Zeichnen. Wiesbaden: Springer Fachmedien Wiesbaden, 2017. doi: 10.1007/978-3-658-18313-4.\\

$[4]$   «Grundlagen der Konstruktionslehre - Conrad - 7.Auflage.pdf».\\

$[5]$   Dokumentation Masterarbeit von Tobias Grätzer.\\

$[6]$   Software Engineering, Ian Sommerville, 2012\\

$[7]$   Gang of Four - Design Patterns\\

$[8]$   Youtube video (Zotero): Anti wind-up\\

$[9]$   RS485 Definition und Beschreibung --> https://www.kunbus.de/rs-485\\

$[10]$  https://de.wikipedia.org/wiki/Framework\\

$[11]$  The Art of Multiprocessor Programming\\

$[12]$  Wikipedia - https://de.wikipedia.org/wiki/Framework

$[13]$  Avast - https://www.avast.com/de-de/c-how-to-check-cpu-temperature

$[14]$  Projektbeschreibung Leroy Harreh

$[15]$  Reichelt Elektronics Webpage

$[16]$ Pandas - https://de.wikipedia.org/wiki/Pandas_(Software)

$[17]$ Numpy - https://de.wikipedia.org/wiki/NumPy

$[18]$ Galileo computing