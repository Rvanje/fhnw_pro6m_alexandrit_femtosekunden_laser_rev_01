\section{Hardware}
In diesem Kapitel wird die Optimierung der Glasefaser behandelt.

\subsection{Laser-Glasfaser Eigenschaften}
Das Material durch das sich die elektromagnetische Welle bewegt, hat im Raum unterschiedliche \textit{refractive indexes}. Entlang diesen Achsen (refractive), bewegt sich das Licht unterschiedlich schnell durch das Material. Z.B. X-polarisierte Wellen bewegen sich schneller durch das Material, wenn der refractive index in der x-Achse des Materials klein ist. In diesem Fall würde die Wellenlänge entlang der X-Achse länger werden, als entlang der Y-Achse.

\lstdefinestyle{custompython}{
  belowcaptionskip=1\baselineskip,
  breaklines=true,
  frame=L,
  xleftmargin=\parindent,
  numbers=left,
  language=Python,
  showstringspaces=false,
  basicstyle=\footnotesize\ttfamily,
  keywordstyle=\bfseries\color{green!40!black},
  commentstyle=\itshape\color{purple!40!black},
  identifierstyle=\color{blue},
  stringstyle=\color{orange},
}

%\lstinputlisting[caption=Ein kurzes Codebeispiel in der Programmiersprache Python, style=custompython]{listings/python_example.py}
