\section{Schlussbemerkungen}
Der Diodentreiber konnte nicht vollständig in Betrieb genommen werden. Der Diodentreiber ist vorhanden und in die Steuerung eingebaut, konnte jedoch nicht fertig getestet werden. Die Ansteuerung erfolgt analog, dies funktioniert jedoch nicht einwandfrei. Am Ausgang konnte eine Spannung gemessen werden, und auch ein Strom konnte gemessen und verändert werden. Das Einstellen des Stromflusses erfolgt über 15 Schritte von 0A-1.5A, von jeweils 0.1A Schritten. Beim Testen konnte immer nur bis zum zweiten Schritt gemessen werden, danach veränderte sich die Anzeige der Messung nicht mehr. Wurde die Spannung erhöht oder verringert, stieg bzw. verringerte sich der Stromfluss wieder. Dies entspricht zwar dem Strom-Spannung-Gesetz, konnte in der entsprechenden Zeit jedoch nicht behoben werden. Es hätte mehr Zeit benötigt, den Treiber besser kennen zu verstehen.\\

% Diese Modi werden eingestellt, indem auf der Platine Kontakte zusammen gelötet werden. Soll ein anderer Modus getestet werden, muss die bestehende Lötstelle aufgelöst werden und eine Andere zusammen gelötet werden. Zusätzlich besitzt der Diodentreiber keinerlei Indikator, wie eine LED, die den Status des Treibers anzeigt. Dies erschwert das Testen des Diodentreibers massiv. Der Laser kann somit nicht angesteuert werden. 

Die Zeit reichte auch nicht aus, um einen Notaus-Taster und das Steuerungskabel zum Laseraufbau einzubauen. Der Notaus-Taster kam im Verlauf der Konstruktion der Steuerung hinzu und fand in der Ausarbeitungsphase keinen Platz mehr. Das Steuerungskabel konnte noch nicht angeschlossen werden, weil mit der Steuerung immer noch Tests durchgeführt wurden. Das Kabel hätte das Testen massiv erschwert, weil dies die Verbindungen nach Aussen fixiert hätte.
% Der Diodentreiber liefert keine 30V, leider wurde davon ausgegangen, dass der Diodentreiber einen DC/DC Wandler \textit{on-board} hat, was so nicht der Fall ist. Es muss ein zweites Netzteil organisiert werden, das 30V liefert.\\a

Der Quellcode der Software funktioniert einwandfrei, trotzdem ist der Code unübersichtlich. Es wäre wünschenswert gewesen, wenn die Objektorientierung im gesamten Code hätte angewendet werden können. Die graphische Benutzeroberfläche beinhaltet einige Details wie Druckknöpfe, Rahmen und Texte die jeweils als ein Objekt hätten dargestellt werden können. Diese hätten einwandfrei in Klassen beschrieben und erzeugt werden können. Das \textit{Multithreading} verhinderte dies jedoch. Mit einer anderen Struktur müsste die trotzdem funktionieren. So wurde die Hauptfunktion des Hauptprogramms 771 Zeilen lang. Die oben genannten Angezeigten Komponenten werden in dieser Funktion alle einzeln erstellt, positioniert und mit Eigenschaften versehen, was die erhöhte Anzahl Zeilen erklärt. Ersichtlich ist dies im Anhang im Kap. \ref{main_src}, Zeile 23 bis hin zur 791. Zeile ist der Konstruktor \textit{def \_\_init\_\_(self):} zu sehen. [18]\\

Daneben wäre für die Parallelisierung der Software, das asynchrone Programmieren vielleicht einfacher gewesen. Die Anwendung der Bibliothek ist wesentlich übersichtlicher und erzielt das selbe Ergebnis.\\

% Der Aufbau des Quellcodes ist ab einem Punkt schwierig zu überblicken. Es war eine gute Entscheidung den Quellcode mit \textit{Git} einer Software zur Versionierung einzusetzen. 

Das Testen der einzelnen Komponenten hat einiges mehr Zeit in Anspruch genommen als Eingangs erwartet. Nahezu jedes Bauteil ist enorm vielfältig und kann in verschiedenen Modi verwendet werden. Dies eröffnete die Möglichkeit Ideen die im Vorfeld bestanden, zu überdenken und die Steuerung anders aufzubauen. So wurde der Diodentreiber des Herstellers \textit{Optlaser} verwendet, anstatt der von \textit{Mainman}. Die Ausgangsleistung von \textit{Optlaser} ist geeigneter, der von \textit{Maiman} mit einer sechsfach höheren Leistung zu hoch.\\

Für eine höhere Leistung des Rechners kann eine neuere Version des \textit{Raspberry PI}s verwendet werden. Diese ermöglichen das Anpassen der Leistungsparameter des Rechners, so kann zum Beispiel der Arbeitsspeicher drastisch erhöht werden.