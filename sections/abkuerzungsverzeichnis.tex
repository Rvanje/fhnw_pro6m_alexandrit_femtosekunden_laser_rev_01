\section*{Abkürzungsverzeichnis}

\begin{table}[H]
    \begin{tabular}{l|l}
         \textbf{Abkürzung}& \textbf{Beschreibung}\\
         \hline
         BUS&   Eine Topologie zur Übertragung von Informationen.\\         
         GUI&   \textit{Graphical User Interface} (dt. Die graphische Benutzeroberfläche)\\
         LDD&   \textit{Laser Diode Driver} (dt. Laserdiodentreiber)\\
         PWM&   Puls Weiten Modulation\\         
         SPS&   Speicher Programmierbare Steuerung\\
         TEC&   \textit{Thermo electric cooler} (dt. Thermoelektrischer Kühler)\\
    \end{tabular}
    % \caption{Abkürzungen}
    \label{tab:abkuerzungen}
\end{table}

\section*{Begriffserklärungen}
\begin{table}[H]
    \begin{tabular}{l|l}
         \textbf{Begriff}& \textbf{Begriffsbeschreibung}\\
         \hline
         Backend&               Die Logik im Hintergrund, die einerseits die Ein- und Ausgaben vom Fron-\\
         &                      tend verarbeitet andererseits Prozesse die gänzlich im Hintergrund ablaufen\\
         &                      auswertet.\\
         Framework&             Ein Framework ist selbst noch kein fertiges Programm, sondern stellt den\\
         &                      Rahmen zur Verfügung, innerhalb dessen der Programmierer eine An-\\
         &                      wendung erstellt. $[12]$\\
         Frontend&              Die Programmierung für die \textit{GUI}.\\
         Image&                 Eine Abbildung eines Betriebssystems mit allen Informationen und der ge-\\
         &                      samten Struktur für einen bestimmten Zweck, der der Rechner erfüllen soll.\\     
         Primary-key&           Die Spalte in einer Datenbank, die den Eintrag (Reihe/Zelle) einzigartig\\
         &                      macht.\\
         Prozessor&             Das Gehirn eines Computers, es besteht grundsätzlich aus mehreren Kernen\\
         &                      in denen Programme ausgeführt werden. Diese Kerne können Informationen\\
         &                      nur seriell verarbeiten.\\
         Raspberry PI&          Einplatinencomputer mit Betriebssystem\\
         Web-basiert&           Eine Applikation, die in einem Webbrowser verwaltbar und nutzbar ist\\
    \end{tabular}
    % \caption{Begriffserklärungen}
    \label{tab:begriffserklaerungen}
\end{table}